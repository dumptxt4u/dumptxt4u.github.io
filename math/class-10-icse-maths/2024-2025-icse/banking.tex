\documentclass[12pt]{article}
\usepackage{amsmath}
\usepackage{enumitem}
\usepackage{multicol}
\usepackage{fancyhdr}
\usepackage{textcomp}
\usepackage[a4paper, top=0.5cm, bottom=1.5cm, left=1.5cm, right=1.5cm]{geometry} % Minimize page padding

\pagestyle{fancy}
\fancyhf{} % Clear all header and footer fields
\renewcommand{\headrulewidth}{0pt} % Remove header rule
\renewcommand{\footrulewidth}{0.4pt} % Add a rule to the footer
\fancyfoot[C]{Trivandrum Maths and Computer Tuition - 9037030309, 04712723471}

\begin{document}
	\section*{Banking ICSE}
	\begin{multicols}{2}
		\section*{Section A}
		\textbf{Multiple Choice Questions}
		
		Choose the correct answer from the given four options (1 to 3):
		
		1. If Sharukh opened a recurring deposit account in a bank and deposited Rs 800 per month for \( 1\frac{1}{2} \) years, then the total money deposited in the account is
		\begin{enumerate}[label=(\alph*)]
			\item[(a)] Rs 11400
			\item[(b)] Rs 14400
			\item[(c)] Rs 13680
			\item[(d)] None of these
		\end{enumerate}
		
		2. Mrs. Asha Mehta deposits Rs 250 per month for one year in a bank's recurring deposit account. If the rate of (simple) interest is 8\% per annum, then the interest earned by her on this account is
		\begin{enumerate}[label=(\alph*)]
			\item[(a)] Rs 65
			\item[(b)] Rs 120
			\item[(c)] Rs 130
			\item[(d)] Rs 260
		\end{enumerate}
		
		3. Mr. Sharma deposited Rs 500 every month in a cumulative deposit account for 2 years. If the bank pays interest at the rate of 7\% per annum, then the amount he gets on maturity is
		\begin{enumerate}
			\item[(a)] Rs 875
			\item[(b)] Rs 6875
			\item[(c)] Rs 10875
			\item[(d)] Rs 12875
		\end{enumerate}
		
		\section*{Section B}
		\begin{enumerate}[label=(\alph*)]
			\item Mr. Dhruv deposits Rs 600 per month in a recurring deposit account for 5 years at the rate of 10\% per annum (simple interest). Find the amount he will receive at the time of maturity.
			
			\item Ankita started paying Rs 400 per month in a 3-year recurring deposit. After six months, her brother Anshul started paying Rs 500 per month in a \( 2\frac{1}{2} \)-year recurring deposit. The bank paid 10\% p.a. simple interest for both. At maturity, who will get more money and by how much?
			
			\item Shilpa has a 4-year recurring deposit account in Bank of Maharashtra and deposits Rs 800 per month. If she gets Rs 48200 at the time of maturity, find:
			\begin{enumerate}
				\item[(i)] the rate of (simple) interest.
				\item[(ii)] the total interest earned by Shilpa.
			\end{enumerate}
			
			\item Mr. Chaturvedi has a recurring deposit account in Grindlay's Bank for \( 4\frac{1}{2} \) years at 11\% p.a. (simple interest). If he gets Rs 101418.75 at the time of maturity, find the monthly instalment.
			
			\item Rajiv Bhardwaj has a recurring deposit account in a bank of Rs 600 per month. If the bank pays simple interest of 7\% p.a. and he gets Rs 15450 as the maturity amount, find the total time for which the account was held.
		\end{enumerate}
	\end{multicols}
	
	\center{******************}
	
\end{document}
