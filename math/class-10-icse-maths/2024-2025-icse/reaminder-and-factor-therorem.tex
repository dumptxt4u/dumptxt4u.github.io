\documentclass[12pt]{article}
\usepackage{amsmath}
\usepackage{enumitem}
\usepackage{textcomp}
\usepackage{multicol}
\usepackage{fancyhdr}
\usepackage[a4paper, top=0.5cm, bottom=1.5cm, left=1.5cm, right=1.5cm]{geometry} % Minimize page padding
\pagestyle{fancy}
\fancyhf{} % Clear all header and footer fields
\renewcommand{\headrulewidth}{0pt} % Remove header rule
\renewcommand{\footrulewidth}{0.4pt} % Add a rule to the footer
\fancyfoot[C]{Trivandrum Maths and Computer Tuition - 9037030309, 04712723471}
\begin{document}
	\begin{multicols}{2}
		\section*{Section A}
	
	\begin{enumerate}
	\item When $2x^3 - x^2 - 3x + 5$ is divided by $2x + 1$, then the remainder is
	\begin{enumerate}
	\item[(a)] 6
	\item[(b)] -6
	\item[(c)] -3
	\item[(d)] 0
	\end{enumerate}
	\item If on dividing $4x^2 - 3kx + 5$ by $x + 2$, the remainder is -3 then the value of k is
	\begin{enumerate}
	\item[(a)] 4
	\item[(b)] -4
	\item[(c)] 3
	\item[(d)] -3
	\end{enumerate}
	\item If on dividing $2x^3 + 6x^2 - (2k - 7)x + 5$ by $x + 3$, the remainder is $k - 1$ then the value of k is
	\begin{enumerate}
	\item[(a)] 2
	\item[(b)] -2
	\item[(c)] -3
	\item[(d)] 3
	\end{enumerate}
	\item If $x + 1$ is a factor of $3x^3 + kx^2 + 7x + 4$, then the value of k is
	\begin{enumerate}
	\item[(a)] -1
	\item[(b)] 0
	\item[(c)] 6
	\item[(d)] 10
	\end{enumerate}
	\section*{Section B}
	\begin{enumerate}
	\item Find the remainder when \( 2x^3 - 3x^2 + 4x + 7 \) is divided by:
	\begin{enumerate}
	\item \( x - 2 \)
	\item \( x + 3 \)
	\item \( 2x + 1 \)
	\end{enumerate}
	
	\item When \( 2x^3 - 9x^2 + 10x - p \) is divided by \( (x + 1) \), the remainder is \(-24\). Find the value of \( p \).
	
	\item If \( (2x - 3) \) is a factor of \( 6x^2 + x + a \), find the value of \( a \). With this value of \( a \), factorise the given expression.
	
	\item When \( 3x^2 - 5x + p \) is divided by \( (x - 2) \), the remainder is 3. Find the value of \( p \). Also, factorise the polynomial \( 3x^2 - 5x + p - 3 \).
	
	\item Prove that \( (5x + 4) \) is a factor of \( 5x^3 + 4x^2 - 5x - 4 \). Hence, factorise the given polynomial completely.
	
	\item Use the factor theorem to factorise the following polynomials completely:
	\begin{enumerate}
	\item \( 4x^3 + 4x^2 - 9x - 9 \)
	\item \( x^3 - 19x - 30 \)
	\end{enumerate}
	
	\item If \( x^3 - 2x^2 + px + q \) has a factor \( (x + 2) \) and leaves a remainder 9 when divided by \( (x + 1) \), find the values of \( p \) and \( q \). With these values of \( p \) and \( q \), factorise the given polynomial completely.
	
	\item If \( (x + 3) \) and \( (x - 4) \) are factors of \( x^3 + ax^2 - bx + 24 \), find the values of \( a \) and \( b \). With these values of \( a \) and \( b \), factorise the given expression.
	
	\item If \( (2x + 1) \) is a factor of both the expressions \( 2x^2 - 5x + p \) and \( 2x^2 + 5x + q \), find the values of \( p \) and \( q \). Hence, find the other factors of both the polynomials.
	
	\item If a polynomial \( f(x) = x^4 - 2x^3 + 3x^2 - ax - b \) leaves remainders 5 and 19 when divided by \( (x - 1) \) and \( (x + 1) \), find the values of \( a \) and \( b \). Hence, determine the remainder when \( f(x) \) is divided by \( (x - 2) \).
	
	\item When a polynomial \( f(x) \) is divided by \( (x - 1) \), the remainder is 5, and when it is divided by \( (x - 2) \), the remainder is 7. Find the remainder when it is divided by \( (x - 1)(x - 2) \).
	\end{enumerate}

	\center{******************}	
	\end{enumerate}
	\end{multicols}
	
\end{document}
