\documentclass[12pt]{article}
\usepackage{amsmath}
\usepackage{enumitem}
\usepackage{multicol}
\usepackage{fancyhdr}
\usepackage{textcomp}
\usepackage[a4paper, top=0.5cm, bottom=1.5cm, left=1.5cm, right=1.5cm]{geometry} % Minimize page padding

\pagestyle{fancy}
\fancyhf{} % Clear all header and footer fields
\renewcommand{\headrulewidth}{0pt} % Remove header rule
\renewcommand{\footrulewidth}{0.4pt} % Add a rule to the footer
\fancyfoot[C]{Trivandrum Maths and Computer Tuition - 9037030309, 04712723471}

\begin{document}
		\begin{multicols}{2}
	\section*{Equation Of Lines ICSE}
\section*{A. Multiple Choice Questions}

Choose the correct option:

\begin{enumerate}
	\item The slope of a line whose angle of inclination is \( 30^\circ \) is:
	\begin{itemize}
		\item[(a)] \( \sqrt{3} \)
		\item[(b)] \( \frac{1}{\sqrt{3}} \)
		\item[(c)] \( -\frac{1}{\sqrt{3}} \)
		\item[(d)] \( -\sqrt{3} \)
	\end{itemize}
	
	\item The angle of inclination of a line having slope 1 is:
	\begin{itemize}
		\item[(a)] \( 30^\circ \)
		\item[(b)] \( 45^\circ \)
		\item[(c)] \( 60^\circ \)
		\item[(d)] \( 90^\circ \)
	\end{itemize}
	
	\item The slope of the line passing through the points \( (0, 4) \) and \( (6, 2) \) is:
	\begin{itemize}
		\item[(a)] 0
		\item[(b)] 1
		\item[(c)] -1
		\item[(d)] 6
	\end{itemize}
	
	\item The slope of the line passing through the points \( (3, 2) \) and \( (-7, -2) \) is:
	\begin{itemize}
		\item[(a)] 0
		\item[(b)] 1
		\item[(c)] -1
		\item[(d)] not defined
	\end{itemize}
	
	\item The slope of a line parallel to the \( y \)-axis is:
	\begin{itemize}
		\item[(a)] 0
		\item[(b)] 1
		\item[(c)] -1
		\item[(d)] not defined
	\end{itemize}
	
	\item The slope of a line parallel to the \( x \)-axis is:
	\begin{itemize}
		\item[(a)] 0
		\item[(b)] 1
		\item[(c)] -1
		\item[(d)] not defined
	\end{itemize}
	
	\item The slope of the line passing through the points \( (3, 2) \) and \( (3, 4) \) is:
	\begin{itemize}
		\item[(a)] -2
		\item[(b)] 0
		\item[(c)] 1
		\item[(d)] not defined
	\end{itemize}
	
	\item The angle of inclination of the line \( y = \frac{1}{\sqrt{3}}x - 5 \) is:
	\begin{itemize}
		\item[(a)] \( 0^\circ \)
		\item[(b)] \( 30^\circ \)
		\item[(c)] \( 45^\circ \)
		\item[(d)] \( 60^\circ \)
	\end{itemize}
	
	\item If the slope of the line passing through the points \( (5, 2) \) and \( (3, k) \) is 2, then the value of \( k \) is:
	\begin{itemize}
		\item[(a)] -1
		\item[(b)] -2
		\item[(c)] -3
		\item[(d)] -6
	\end{itemize}
	
	\item The slope of a line parallel to the line passing through the points \( (6, 0) \) and \( (-3, 7) \) is:
	\begin{itemize}
		\item[(a)] \( \frac{7}{9} \)
		\item[(b)] \( -\frac{7}{9} \)
		\item[(c)] \( \frac{9}{7} \)
		\item[(d)] \( -\frac{9}{7} \)
	\end{itemize}
	
	\item The slope of a line perpendicular to the line passing through the points \( (2, 5) \) and \( (-3, 6) \) is:
	\begin{itemize}
		\item[(a)] 5
		\item[(b)] -5
		\item[(c)] \( \frac{1}{5} \)
		\item[(d)] \( -\frac{1}{5} \)
	\end{itemize}
	
	\item The slope of a line parallel to the line \( 3x + 2y - 7 = 0 \) is:
	\begin{itemize}
		\item[(a)] \( -\frac{2}{3} \)
		\item[(b)] \( \frac{2}{3} \)
		\item[(c)] \( -\frac{3}{2} \)
		\item[(d)] \( \frac{3}{2} \)
	\end{itemize}
	
	\item The slope of the line \( x - 2y = 1 \) is:
	\begin{itemize}
		\item[(a)] 0
		\item[(b)] 1
		\item[(c)] \( \frac{1}{2} \)
		\item[(d)] \( -\frac{1}{2} \)
	\end{itemize}
	
	\item The angle of inclination of the line \( \sqrt{3}x - y = 1 \) is:
	\begin{itemize}
		\item[(a)] \( 30^\circ \)
		\item[(b)] \( 45^\circ \)
		\item[(c)] \( 60^\circ \)
		\item[(d)] \( 90^\circ \)
	\end{itemize}
	
	\item The equation of the line whose inclination is \( 45^\circ \) and which intersects the \( y \)-axis at the point \( (0, -4) \) is:
	\begin{itemize}
		\item[(a)] \( x - y = 4 \)
		\item[(b)] \( x + y = 4 \)
		\item[(c)] \( y - x = 4 \)
		\item[(d)] \( x - y = -4 \)
	\end{itemize}
	
	\item If the point \( (a, 2a) \) lies on the line \( y = 3x - 6 \), then the value of \( a \) is:
	\begin{itemize}
		\item[(a)] 1
		\item[(b)] 3
		\item[(c)] 6
		\item[(d)] 4
	\end{itemize}
\end{enumerate}
		\section*{Section B}
	\begin{enumerate}
		\item What is the value of \( x \) so that the line through \( (4, 1) \) and \( (6, 2) \) is perpendicular to the line joining \( (x, 2) \) and \( (4, 6) \)?
		\item What is the value of \( a \) so that the line through \( (a, 0) \) and \( (3, 2) \) is perpendicular to the line joining \( (1, 2) \) and \( (-6, 1) \)?
		\item Without using Pythagoras theorem, show that the following points are the vertices of a right-angled triangle: \( D(0, 4) \), \( E(1, 2) \), \( F(3, 3) \).
		\item Find the equation of a line that is equidistant from the lines \( x = 5 \) and \( x = 3 \).
		\item If \( 2x + 3 = \frac{p}{2}x + 3 \) are parallel, find the value of \( p \).
		\item If \( (2a + 1)x + 3 = 0 \) and \( 8y - (2 - 1)x = 5 \) are perpendicular to each other, find \( a \).
		\item Find the equation of a line that has \( y \)-intercept \(-4\) and is parallel to the line joining \( (2, -5) \) and \( (1, 2) \).
		\item Find the equation of a line that has \( y \)-intercept \(-6\) and is perpendicular to the line joining \( (-1, 6) \) and \( (-2, 4) \).
		\item Find the equations of the straight lines passing through the points \( (2, 3) \) and \( (4, 1) \).
		\item In what ratio does the line joining the points \( (2, 3) \) and \( (4, -5) \) divide the line passing through the points \( (6, 8) \) and \( (1, -1) \)?
		\item Find the value of \( p \), given that the line through the point \( (-4, 4) \) and \( \frac{y}{2} = x - p \) passes.
		\item Find the value of \( m \), given that the line \( 2mx - 5y + 13 = 0 \) passes through the point \( (-1, 2) \).
		\item The graph of the equation \( y = mx + c \) passes through the points \( (1, 4) \) and \( (-2, 5) \). Find \( m \) and \( c \).
		\item Points \( A \) and \( B \) have coordinates \( (7, 3) \) and \( (1, 9) \), respectively. Find:
		\begin{enumerate}
			\item The slope of \( AB \).
			\item The equation of the perpendicular bisector of the line segment \( AB \).
			\item The value of \( p \) if \( (-2, p) \) lies on the bisector.
		\end{enumerate}
		\item The side \( AB \) of a rectangle \( ABCD \) is parallel to the \( y \)-axis. Calculate:
		\begin{enumerate}
			\item The slope of \( AD \),
			\item The slope of \( BD \),
			\item The slope of \( AC \).
		\end{enumerate}
		\item If \( A(-3, -4) \), \( B(2, 6) \), and \( C(-6, 10) \) are the vertices of a triangle \( ABC \), find the equation of the median through \( A \).
		\item Find the equations of the altitudes of the triangle whose vertices are given as \( (10, 4) \), \( (-4, 9) \), and \( (-2, -1) \).
		\item Find the equations of the sides of the triangle whose angular points are given as \( (-1, 2) \), \( (6, 0) \), and \( (2, 5) \).
		\item Find the equation of the straight line that passes through the point \( (3, 4) \) and is perpendicular to the line \( 3x + 2y + 5 = 0 \).
		\item Write down the equation of the line \( AB \), through \( (3, 2) \) and perpendicular to the line \( 2y = 3x + 5 \). If \( AB \) meets the \( x \)-axis at \( A \) and \( y \)-axis at \( B \), write the coordinates of \( A \) and \( B \). Calculate the area of triangle \( OAB \), where \( O \) is the origin.
		\item Find the equation of the line parallel to \( 3x - 4y + 6 = 0 \) and passing through the midpoint of the segment joining \( (2, 3) \) and \( (4, -1) \).
		\item Write down the equation of the line whose slope is \(-1\) and which passes through \( P \), where \( P \) divides the line segment joining \( A(-1, 2) \) and \( B(3, 6) \) in the ratio \( 1:3 \).
		\item The points \( A(7, 3) \) and \( C(0, -4) \) are two opposite vertices of a rhombus \( ABCD \). Find the equation of the diagonal \( BD \).
		\item Find the equation of the line passing through the points \( (-1, 2) \) and the point of intersection of the lines \( 6x - 5y + 2 = 0 \) and \( 5x - 6y + 9 = 0 \).
		\item Find the equation of the line which makes equal intercepts on the axes and passes through the point \( (2, 3) \).
		\item \( P(3, 4) \), \( Q(7, -2) \), and \( R(-2, -1) \) are the vertices of triangle \( PQR \). Write down the equation of the median of the triangle through \( R \).
	
	 \item A straight line passes through the points \( P(-1, 4) \) and \( Q(5, -2) \). It intersects the coordinate axes at points \( A \) and \( B \). \( M \) is the midpoint of the segment \( AB \).
	\begin{enumerate}
		\item[(i)] Find the equation of the line.
		\item[(ii)] Find the coordinates of \( A \) and \( B \).
		\item[(iii)] Find the coordinates of \( M \).
	\end{enumerate}
	
	\item Find the equation of the line which passes through the point \( (2, 6) \) and is such that the intercept on the \( x \)-axis exceeds the intercept on the \( y \)-axis by 5.
	
	\end{enumerate}
	\end{multicols}
	\center{******************}
	
\end{document}
