\documentclass{article}
\usepackage[margin=1in]{geometry}
\usepackage{multicol}
\usepackage{listings}
\usepackage{graphicx}
\usepackage{amsmath}
\usepackage{booktabs}

\title{Chapter 2: Class as the Basis of All Computations \\ Extracted Questions from PDF}
\author{}
\date{}

\begin{document}
	\maketitle
	
	\begin{multicols}{2}
		
		\section{Multiple Choice Questions (1--20)}
		
		\subsection{1. Which of the following is NOT a feature of Object-Oriented Programming (OOP) implemented by classes in Java?}
		\begin{enumerate}
			\renewcommand{\labelenumi}{(\alph{enumi})}
			\item Abstraction
			\item Inheritance
			\item Encapsulation
			\item Compilation
		\end{enumerate}
		
		\subsection{2. What is the primary purpose of the new operator in Java?}
		\begin{enumerate}
			\renewcommand{\labelenumi}{(\alph{enumi})}
			\item To declare a variable
			\item To allocate memory for an object at runtime
			\item To define a class
			\item To initialize a constant
		\end{enumerate}
		
		\subsection{3. Which of the following correctly declares and initializes an object in Java?}
		\begin{enumerate}
			\renewcommand{\labelenumi}{(\alph{enumi})}
			\item Student student1;
			\item Student1 = Student();
			\item Student student1 = new Student();
			\item New Student = student1();
		\end{enumerate}
		
		\subsection{4. Which access specifier allows members to be accessible within the same class, subclasses, and other classes in the same package?}
		\begin{enumerate}
			\renewcommand{\labelenumi}{(\alph{enumi})}
			\item Private
			\item Public
			\item Protected
			\item Default
		\end{enumerate}
		
		\subsection{5. In Java, what is the default value of a boolean variable?}
		\begin{enumerate}
			\renewcommand{\labelenumi}{(\alph{enumi})}
			\item Null
			\item 0
			\item True
			\item False
		\end{enumerate}
		
		\subsection{6. Which type of variable is declared within a method and is destroyed once the method is finished?}
		\begin{enumerate}
			\renewcommand{\labelenumi}{(\alph{enumi})}
			\item Instance Variable
			\item Local Variable
			\item Static Variable
			\item Final Variable
		\end{enumerate}
		
		\subsection{7. What is the size of an int data type in Java?}
		\begin{enumerate}
			\renewcommand{\labelenumi}{(\alph{enumi})}
			\item 1 byte
			\item 2 bytes
			\item 4 bytes
			\item 8 bytes
		\end{enumerate}
		
		\subsection{8. What is the primary feature of a final variable in Java?}
		\begin{enumerate}
			\renewcommand{\labelenumi}{(\alph{enumi})}
			\item It can store the reference of multiple objects
			\item Its value can be modified during runtime
			\item Its value cannot be changed once assigned
			\item It is accessible only within the class
		\end{enumerate}
		
		\subsection{9. What does the following line of code do? Student student1 = new Student();}
		\begin{enumerate}
			\renewcommand{\labelenumi}{(\alph{enumi})}
			\item Declares a class named Student
			\item Creates a new object of the Student class and assigns its reference to student1
			\item Declares a local variable named Student
			\item Assigns a new class to the variable student1
		\end{enumerate}
		
		\subsection{10. What is the primary purpose of encapsulation in Java?}
		\begin{enumerate}
			\renewcommand{\labelenumi}{(\alph{enumi})}
			\item To enable inheritance between classes
			\item To hide the implementation details and protect data
			\item To provide faster compilation
			\item To improve runtime efficiency
		\end{enumerate}
		
		\subsection{11. A class is called :}
		\begin{enumerate}
			\renewcommand{\labelenumi}{(\alph{enumi})}
			\item Object Factory
			\item User defined data type
			\item Composite data type
			\item All of these
		\end{enumerate}
		
		\subsection{12. Which of the following is not a keyword?}
		\begin{enumerate}
			\renewcommand{\labelenumi}{(\alph{enumi})}
			\item Void
			\item Byte
			\item Object
			\item Public
		\end{enumerate}
		
		\subsection{13. A variable whose value exists throughout the life of the program is a?}
		\begin{enumerate}
			\renewcommand{\labelenumi}{(\alph{enumi})}
			\item Local variable
			\item Static variable
			\item Final variable
			\item None of these
		\end{enumerate}
		
		\subsection{14. Which operator is used to allocate memory to an object?}
		\begin{enumerate}
			\renewcommand{\labelenumi}{(\alph{enumi})}
			\item Dot
			\item New
			\item Both (a) and (b)
			\item None of these
		\end{enumerate}
		
		\subsection{15. The characteristics of a class are represented by which variable?}
		\begin{enumerate}
			\renewcommand{\labelenumi}{(\alph{enumi})}
			\item Local variables
			\item Instance variables
			\item Static variables
			\item Final variables
		\end{enumerate}
		
		\subsection{16. In the statement Student stu = new Student(); what is the name of the object?}
		\begin{enumerate}
			\renewcommand{\labelenumi}{(\alph{enumi})}
			\item Student
			\item stu
			\item Stu
			\item None of these
		\end{enumerate}
		
		\subsection{17. In the below class, choose the name of the data members :}
		class Test \{ String name; int age; public static void main() \{ float x = 5; int p; \} \}
		\begin{enumerate}
			\renewcommand{\labelenumi}{(\alph{enumi})}
			\item name, age
			\item x, p
			\item name, x
			\item age, p
		\end{enumerate}
		
		\subsection{18. For which of the following data types, the size of the variable declared will be fixed?}
		\begin{enumerate}
			\renewcommand{\labelenumi}{(\alph{enumi})}
			\item Primitive
			\item Reference
			\item Composite
			\item None of these
		\end{enumerate}
		
		\subsection{19. A class implements which of the following OOP characteristics?}
		\begin{enumerate}
			\renewcommand{\labelenumi}{(\alph{enumi})}
			\item Encapsulation
			\item Inheritance
			\item Abstraction
			\item Polymorphism
		\end{enumerate}
		
		\subsection{20. The messages are passed in a class using :}
		\begin{enumerate}
			\renewcommand{\labelenumi}{(\alph{enumi})}
			\item Local variables
			\item Function parameters
			\item Objects
			\item None of these
		\end{enumerate}
		
		\section{Short Answer Questions (51--65)}
		
		\subsection{51. What does a class encapsulate?}
		\subsection{52. What are the different set of data types available in Java?}
		\subsection{53. Difference between public and private modifiers for members of a class.}
		\subsection{54. List various reference data types of Java.}
		\subsection{55. Write a Java statement to compute discount.}
		\subsection{56. What is the difference between an object and a class?}
		\subsection{57. Why is a class called a factory of objects?}
		\subsection{58. Define Instance Variable with example.}
		\subsection{59. Assign the value of pie (3.142) to a variable.}
		\subsection{60. What is a class in Java?}
		\subsection{61. What are the features of OOP implemented by classes?}
		\subsection{62. What does an object encapsulate in Java?}
		\subsection{63. What are the primary access specifiers in Java?}
		\subsection{64. Differentiate between primitive and reference data types.}
		\subsection{65. Explain the steps for declaring and initializing an object in Java.}
		
	\end{multicols}
\end{document}
